\documentclass[12pt]{article}
\def\shownotes{1}

\usepackage[top=3cm, bottom=3cm, left=2cm, right=2cm]{geometry}      % [top=2cm, bottom=2cm, left=2cm, right=2cm]
\geometry{letterpaper}                   % ... or a4paper or a5paper or ... 
%\geometry{landscape}                % Activate for for rotated page geometry
 
\usepackage{graphicx}
\usepackage{amssymb, amsmath, amsthm, amsfonts}
\usepackage{enumerate}
\usepackage{hyperref}
\usepackage{xspace}
\usepackage{graphicx}
\usepackage{latexsym}
\usepackage{color}
\usepackage{framed}
\usepackage{algpseudocode} 

\usepackage{mathtools}

\DeclarePairedDelimiter\abs{\lvert}{\rvert}%

\usepackage[normalem]{ulem}
%\usepackage{cancel}

\newcommand{\red}[1]{\textcolor{red}{#1}} 


\mathchardef\mhyphen="2D

\ifnum\shownotes=1
\newcommand{\authnote}[2]{{\textcolor{red}{\textsf{#1 notes: }\textcolor{blue}{ #2}}\marginpar{\textcolor{red}{\textbf{!!!!!}}}}}
\else
\newcommand{\authnote}[2]{}
\fi
\newcommand{\bnote}[1]{{\authnote{Ben}{#1}}}
\newcommand{\lnote}[1]{{\authnote{Leo}{#1}}}
\newcommand{\xnote}[1]{{\authnote{Xianrui}{#1}}}

\newcommand{\class}[1]{{\ensuremath{\mathsf{#1}}}}

\newcommand{\gen}{\ensuremath{\class{Gen}}\xspace}
\newcommand{\rep}{\ensuremath{\class{Rep}}\xspace}

\newcommand{\rec}{\ensuremath{\class{Rec}}\xspace}
\newcommand{\enc}{\ensuremath{\class{Enc}}\xspace}
\newcommand{\dec}{\ensuremath{\class{Dec}}\xspace}
\newcommand{\prg}{\ensuremath{\class{prg}}\xspace}
\newcommand{\zo}{\ensuremath{\{0, 1\}}}
\newcommand{\vect}[1]{\ensuremath{\textbf{#1}}}
\newcommand{\zq}{\ensuremath{\mathbb{Z}_q}}
\newcommand{\Fq}{\ensuremath{\mathbb{F}_q}}
\newcommand{\sample}{\ensuremath{\class{Sample}}\xspace}
\newcommand{\neigh}{\ensuremath{\class{Neigh}}\xspace}
\newcommand{\dis}{\ensuremath{\mathsf{dis}}}
\newcommand{\decode}{\ensuremath{\mathsf{Decode}}}

\newcommand{\ve}{\vect{e}}
\newcommand{\vm}{\vect{m}}
\newcommand{\vy}{\vect{y}}
\newcommand{\vE}{\vect{E}}
\newcommand{\vS}{\vect{S}}
\newcommand{\vA}{\vect{A}}
\newcommand{\vc}{\vect{c}}
\newcommand{\vW}{\vect{W}}
\newcommand{\vR}{\vect{R}}
\newcommand{\vU}{\vect{U}}
\newcommand{\vT}{\vect{T}}
\newcommand{\vX}{\vect{X}}
\newcommand{\vB}{\vect{B}}
\newcommand{\vz}{\vect{z}}
\newcommand{\vq}{\vect{q}}
\newcommand{\vd}{\vect{d}}
\newcommand{\vs}{\vect{s}}
\newcommand{\vx}{\vect{x}}
\newcommand{\va}{\vect{a}}
\newcommand{\vb}{\vect{b}}
\newcommand{\vgamma}{\mathbf{\Gamma}}
\newcommand{\vt}{\vect{t}}
\newcommand{\vu}{\vect{u}}
\newcommand{\vF}{\vect{F}}
\newcommand{\recout}{x}
\newcommand{\ignore}[1]{}


\newcommand{\F}[1]{\mathcal{F}_{\textbf{#1}}}
\newcommand{\A}{\mathcal{A}}
\renewcommand{\L}{\mathcal{L}}
\renewcommand{\S}{\mathcal{S}}
\renewcommand{\O}{\mathcal{O}}


\newcommand{\search}{\class{Search}}
\newcommand{\getrind}{\class{GetRind}}
\newcommand{\edbsetup}{\class{EDBSetup}}
\newcommand{\token}{\class{Token}}
\newcommand{\SSEEnc}{\class{SSEEnc}}
\newcommand{\SSEToken}{\class{Token}}
\newcommand{\Search}{\class{Search}}


\newcommand{\Tsetsetup}{\class{TSetSetup}}
\newcommand{\Tsetgettag}{\class{TSetGetTag}}
\newcommand{\Tsetretrieve}{\class{TSetRetrieve}}
\newcommand{\W}{\class{W}}
\newcommand{\stag}{\class{stag}}
\newcommand{\xtag}{\class{xtag}}
\newcommand{\xind}{\class{xind}}
\newcommand{\xtoken}{\class{xtoken}}
\newcommand{\xtrap}{\class{xtrap}}
\newcommand{\Tset}{\class{TSet}}
\newcommand{\Xset}{\class{XSet}}
\newcommand{\edb}{\class{EDB}}
\newcommand{\db}{\class{DB}}
\newcommand{\sketch}{\ensuremath{\class{Sketch}}\xspace}


\newtheorem{theorem}{Theorem}[section]
\newtheorem{lemma}[theorem]{Lemma}
\newtheorem{proposition}[theorem]{Proposition}
\newtheorem{corollary}[theorem]{Corollary}
\newtheorem{definition}[theorem]{Definition}
\newtheorem{assumption}[theorem]{Assumption}
\newtheorem{claim}[theorem]{Claim}
\newtheorem{problem}[theorem]{Problem}
\newtheorem{construction}[theorem]{Construction}


\newcommand{\ind}{\class{ind}}
\newcommand{\rind}{\class{rind}}

\begin{document}

\begin{itemize}
\item $\SSEEnc(K, G)$:
\begin{enumerate}
\item $G = (V, E)$, for each $u \in V$. Let $r = \lfloor \log V\rfloor$.\\
Sample $r + 1$ many sets of sizes $1, 2, 2^2, ..., 2^r$ respectively. In each set, the samples are chosen uniformly at random from all nodes. Call these sets $S_0, ..., S_r$. 
\item For each node $u\in V$. 
\begin{enumerate}
\item First compute $\kappa_u \leftarrow f_{k_1}(u)$.
\item For all these sets $S_i$, compute $(w_i, \delta_i)$, where $w_i$ is the closest node to $u$ in $S_i$, i.e. $\delta_i = d(u, w_i) = argmin_{v\in S_i}d(u, v)$.
\item Compute $\kappa_{w_i} \leftarrow f_{k_2}(w_i)$, $\kappa_{u}' \leftarrow f_{k_3}(u)$,$tag_i \leftarrow g^{\kappa_{w_i} \cdot \kappa_u'^{-1}}$, $c^u_i \leftarrow Enc_{K_e}(\delta_i)$. \\
Store $\Tset[u] = ((tag_1, c^u_1), ..., (tag_r, c^u_r))$
\end{enumerate}
\item Repeat step (2) $k$ many times. Each time sampling independently at random. Union the $\Tset[u]$ each run of step (2).
\end{enumerate}

\item $\token(K, u, v)$: $t_{u1} \leftarrow f_{k_1}(u)$, $t_{v1} \leftarrow f_{k_1}(v)$, $t_{u2} \leftarrow f_{k_3}(u)$, $t_{v2} \leftarrow f_{k_3}(v)$.\\
Output $t = (t_{u1}, t_{v1}, t_{u2}, t_{v2})$.\\


\item $\search(t, C)$: (Note: this is NOT Bourgain's query)
\begin{enumerate}
\item Parse $t$ as $(t_{u1}, t_{v1}, t_{u2}, t_{v2})$.
\item Search for $\sketch[t_{u1}] = \vt = \{\alpha^u_i, \beta^u_i\}$ and $\sketch[t_{v1}] = \vt = \{\alpha^v_i, \beta^v_i\}$.\\
For each $\alpha_i^u$, compute $w_u^i = (\alpha_i^u)^{t_{u2}}$. For each $\alpha_i^v$, compute $w_v^i = (\alpha_i^v)^{t_{v2}}$
\item Test the common elements in for the set $\{w_u^i\}$ and $\{w_v^i\}$, and retrieve the corresponding $\beta^u_i$ $\beta^v_j$, and compute  $\delta_k = \beta^u_i \beta^v_j$. 
\end{enumerate}

\item $\dec(K, \vc)$: Decrypt the $\dec_{K_e}(\delta_k)$ received from the server, and get the minimum one.
\end{itemize}
\end{document}

